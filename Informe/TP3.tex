\documentclass[A4paper,oneside,fleqn,11pt]{article}

% This first part of the file is called the PREAMBLE. It includes
% customizations and command definitions. The preamble is everything
% between \documentclass and \begin{document}.

%Cambiamos un poquito los márgenes%
\addtolength{\oddsidemargin}{-1in}
\addtolength{\evensidemargin}{-1in}
\addtolength{\textwidth}{2in}
\addtolength{\topmargin}{-1in}
\addtolength{\textheight}{2in}



\usepackage{mathtools}
\usepackage{graphicx}              % to include figures
\usepackage{amsmath}               % great math stuff
\usepackage{amsmath,scalerel}
\usepackage{amsfonts}              % for blackboard bold, etc
\usepackage{amsthm}                % better theorem environments
\usepackage{amssymb}
\usepackage{mathrsfs}
\usepackage[spanish]{babel}
\usepackage[utf8]{inputenc}
\usepackage{hyperref}
\usepackage{multicol}
\usepackage{tikz-cd}
\usepackage{amsmath}
\usepackage[linesnumbered,ruled]{algorithm2e}
\usepackage{algpseudocode}
\usetikzlibrary{calc}
\usetikzlibrary{matrix}
\usepackage{graphicx,wrapfig,lipsum}
\usepackage{subcaption} %para poner varias imagenes en un figure
\usepackage{graphicx} % Required for including pictures

\usepackage{sidecap}%para poner la descripción de una imagen al lado de la imagen,no abajo ni arriba
\usepackage{float} % Allows putting an [H] in \begin{figure} to specify the exact location of the figure
\usepackage{wrapfig} % Allows in-line images such as the example fish picture
\graphicspath{ {Grphs/} }


\setcounter{tocdepth}{3}% to get subsubsections in toc

\let\oldtocsection=\tocsection

\let\oldtocsubsection=\tocsubsection

\let\oldtocsubsubsection=\tocsubsubsection

% various theorems, numbered by section

\newtheorem{teo}{Teorema}[section]
\newtheorem{lem}[teo]{Lema}
\newtheorem{prop}[teo]{Proposición}
\newtheorem{cor}[teo]{Corolario}
\newtheorem{crit}[teo]{Criterio}
\newtheorem{propi}[teo]{Propiedad}

\theoremstyle{definition}
\newtheorem{ejcio}[teo]{Ejercicio}
\newtheorem{conj}[teo]{Conjetura}
\newtheorem{obs}[teo]{Observación}
\newtheorem{defn}[teo]{Definición}
\newtheorem{ax}[teo]{Axioma}
\newtheorem{ex}[teo]{Ejemplo}

\newcommand{\bd}[1]{\mathbf{#1}}  % for bolding symbols
\newcommand{\cl}[1]{\overline{#1}} 
\newcommand{\CC}{\mathbb{C}}
\newcommand{\RR}{\mathbb{R}}      % for Real numbers
\newcommand{\ZZ}{\mathbb{Z}}      % for Integers
\newcommand{\NN}{\mathbb{N}}
\newcommand{\QQ}{\mathbb{Q}}
\newcommand{\FF}{\mathbb{F}}
\newcommand{\col}[1]{\left[\begin{matrix} #1 \end{matrix} \right]}
\newcommand{\comb}[2]{\binom{#1^2 + #2^2}{#1+#2}}
\newcommand{\eps}{\varepsilon}
\renewcommand{\hom}{\mathrm{Hom}}
\let\oldemptyset\emptyset
\let\emptyset\varnothing
\DeclareMathOperator{\id}{id}
\DeclareMathOperator{\mcm}{mcm}
\DeclareMathOperator{\mcd}{mcd}
\DeclareMathOperator{\ord}{ord}
\DeclareMathOperator{\im}{im}
\DeclareMathOperator{\End}{End}
\DeclareMathOperator{\Aut}{Aut}
\DeclareMathOperator{\sg}{sg}
\DeclareMathOperator{\cok}{cok}
\DeclareMathOperator{\ext}{Ext}
\DeclareMathOperator{\Obj}{Obj}
\DeclareMathOperator{\rank}{rk}
\DeclareMathOperator{\gr}{gr}
\DeclareMathOperator{\car}{char}
\DeclareMathOperator{\Nil}{Nil}
\DeclareMathOperator{\spec}{Spec}
\DeclareMathOperator{\ev}{ev}
\DeclareMathOperator{\ann}{Ann}
\DeclareMathOperator{\tr}{Tr}
\DeclareMathOperator*{\bigcdot}{\scalerel*{\cdot}{\bigodot}}
\def\acts{\curvearrowright}
\def\stca{\curvearrowleft}

\setcounter{tocdepth}{10}
\setcounter{secnumdepth}{10}

\usepackage[utf8]{inputenc}
\usepackage{fancyhdr}


\begin{document}

%----------------------------------------------------------------------------------------
%	TITLE PAGE
%----------------------------------------------------------------------------------------

\begin{titlepage}

\center % Center everything on the page

\newcommand{\HRule}{\rule{\linewidth}{0.5mm}} % Defines a new command for the horizontal lines, change thickness here

\textsc{\LARGE Universidad de Buenos Aires}\\[1.5cm] % Name of your university/college
\textsc{\Large Facultad de Ciencias Exactas y Naturales}\\[0.5cm] % Major heading such as course name
\textsc{\large Departamento de Computación}\\[0.5cm] % Minor heading such as course title
\textsc{\large Algoritmos y Estructuras de Datos III}\\[0.5cm] % Minor heading such as course title

\HRule \\[0.8cm]
{ \huge \bfseries Trabajo Práctico 3}\\[0.4cm] % Title of your document
\HRule \\[1.5cm]

\begin{minipage}{0.8\textwidth}
\center
%\begin{flushleft} 
\Large
\emph{Autores:}


{Nicolás Chehebar, mail: \textit{nicocheh@hotmail.com}, LU: 308/16 

Matías Duran, mail: \textit{mato\_ fede@live.com.ar}, LU: 400/16 

Lucas Somacal, mail: \textit{lsomacal@gmail.com}, LU: 249/16} % Your name
%\end{flushleft}
~
\end{minipage}\\[4cm]

%\includegraphics{Logo}\\[1cm] % Include a department/university logo - this will require the graphicx package

\vfill % Fill the rest of the page with whitespace

\end{titlepage}

%----------------------------------------------------------------------------------------
%	TABLE OF CONTENTS
%----------------------------------------------------------------------------------------


 %\chead{Algo III, TP2, Chehebar, Duran, Somacal}
 
%\title{Algoritmos y Estructuras de Datos III, TP2}
%\author{Nicolás Chehebar, Matías Duran, Lucas Somacal}
%\date{}



\pagenumbering{roman}
\pagenumbering{arabic}
%\maketitle
\tableofcontents
\clearpage







\section{El Juego}
\subsection{Descripción}
El juego es una generalización del popular 4 en línea \footnote{$https://es.wikipedia.org/wiki/Conecta_4$}. Consiste en una grilla de $M$ filas y $N$ columnas en la cual dos jugadores colocan una ficha propia (identificada con rojo las de un jugador y azul las del otro) alternadamente. Las fichas se pueden colocar en cualquier columna de la grilla y una vez elegida la columna esta determina el movimiento, ya que irá a la fila de "más abajo" (la de numeración más baja) que este desocupada. El objetivo de un jugador será lograr tener un línea recta (diagonal, vertical u horizontal) de $C$ fichas propias. Cuando esto suceda el jugador ganará el partido. Más aún, cada jugador dispone de $P$ fichas. En caso de que ambos se queden sin fichas y no haya habido ningun jugador, la partida finaliza en empate. También se da un empate si la grilla queda llena y ningun jugador ha ganado. Se trata de una generalización del 4 en línea ya que si tomamos parametros $M=6, N=7, C=4, P=21$ se trata del clasico juego del 4 en línea.
\subsubsection{Ejemplos}




\section{Jugador Óptimo}

\subsection{El algoritmo}

El algoritmo del jugador del punto 1.a brinda un jugador óptimo. Nos asegura que este jugador hará la estrategia ganadora si hubiera. En caso de que esta no exista, realizara una de empate. Y si tampoco existiera esa, jugara indistintamente sabiendo que perderá. Este análisis se realiza jugada a jugada. Para ejemplificar esto, podría suceder que el oponente tenga la estrategia ganadora y en ese caso nuestro jugador jugara cualquier jugada indistintamente (pues sabe perderá), pero si en la próxima jugada el oponente no realiza la correspondiente a su estrategia ganadora y da un nuevo estado del tablero en el que esta vez nuestro jugador tiene estrategia ganadora, nuestro jugador jugará y ganará ya que ahora sí tiene estrategia ganadora.

Para lograr esto, utilizamos una técnica algorítmica similar al Backtracking en el sentido de que exploramos todas las soluciones posibles y nos quedamos con la óptima. Pero esta vez tenemos dos jugadores interviniendo en la situación donde lo que uno busca es todo lo contrario a lo que busca el otro. Podemos decir que un tablero finalizado tiene 3 puntajes posibles, 1 si ganamos nosotros, 0 si es empate, -1 si gano el otro (podríamos sino generalizarlo para todo tablero y que haya un cuarto valor que sea invalido si aun no hemos calculado el valor de dicho tablero). Así, lo que sabemos es que turno a turno, uno quiere maximizar el puntaje y el otro minimizarlo. Es por esto que dicha técnica algorítmica se llama Minimax. 

De esta manera, igual que en Backtracking tenemos un arbol de ejecución donde cada nodo es un estado del tablero y la raiz es el tablero vacío. Cada nodo (que  no sea hoja) tendrá $N$ hijos donde cada uno representará que la próxima jugada fue en alguna de las $N$ columnas. De esta forma, recorremos todos los tableros posibles. Según quien comience, en el primer nivel trataremos de maximizar o minimizar y en el siguiente lo contrario y así sucesivamente. Todos los niveles impares minimizarán si empieza el contrincante y maximizarán si empieza nuestro jugador. El que maximiza le asignará a su nodo un puntaje que será el máximo de los puntajes de todos sus hijos, análogamente el que minimiza le asignará a su nodo un puntaje que será el mínimo de los puntajes de todos sus hijos.

Así, ejecutando dicho algoritmo la raíz tendrá la información de quién tiene la estrategia ganadora, o que ambos pueden asegurar el empate según quien empiece y haya un (1, 0 o -1). Esta es la idea general del algoritmo, lo veremos mas claro en el siguiente pseudocódigo

\subsubsection{El Pseudocódigo}


\subsection{La poda alfa-beta}


\subsubsection{El algoritmo}

\subsubsection{El Pseudocódigo}



\subsection{Complejidad}









\subsection{Experimentación}

\subsubsection{Contexto} 

\subsubsection{Sin poda}

\subsubsection{Con poda}





\section{Análisis comparativo con paper}








\end{document}