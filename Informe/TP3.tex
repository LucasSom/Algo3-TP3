\documentclass[A4paper,oneside,fleqn,11pt]{article}

% This first part of the file is called the PREAMBLE. It includes
% customizations and command definitions. The preamble is everything
% between \documentclass and \begin{document}.

%Cambiamos un poquito los márgenes%
\addtolength{\oddsidemargin}{-1in}
\addtolength{\evensidemargin}{-1in}
\addtolength{\textwidth}{2in}
\addtolength{\topmargin}{-1in}
\addtolength{\textheight}{2in}



\usepackage{mathtools}
\usepackage{graphicx}              % to include figures
\usepackage{amsmath}               % great math stuff
\usepackage{amsmath,scalerel}
\usepackage{amsfonts}              % for blackboard bold, etc
\usepackage{amsthm}                % better theorem environments
\usepackage{amssymb}
\usepackage{mathrsfs}
\usepackage[spanish]{babel}
\usepackage[utf8]{inputenc}
\usepackage{hyperref}
\usepackage{multicol}
\usepackage{tikz-cd}
\usepackage{amsmath}
\usepackage[linesnumbered,ruled]{algorithm2e}
\usepackage{algpseudocode}
\usetikzlibrary{calc}
\usetikzlibrary{matrix}
\usepackage{graphicx,wrapfig,lipsum}
\usepackage{subcaption} %para poner varias imagenes en un figure
\usepackage{graphicx} % Required for including pictures

\usepackage{sidecap}%para poner la descripción de una imagen al lado de la imagen,no abajo ni arriba
\usepackage{float} % Allows putting an [H] in \begin{figure} to specify the exact location of the figure
\usepackage{wrapfig} % Allows in-line images such as the example fish picture
\graphicspath{ {Grphs/} }


\setcounter{tocdepth}{3}% to get subsubsections in toc

\let\oldtocsection=\tocsection

\let\oldtocsubsection=\tocsubsection

\let\oldtocsubsubsection=\tocsubsubsection

% various theorems, numbered by section

\newtheorem{teo}{Teorema}[section]
\newtheorem{lem}[teo]{Lema}
\newtheorem{prop}[teo]{Proposición}
\newtheorem{cor}[teo]{Corolario}
\newtheorem{crit}[teo]{Criterio}
\newtheorem{propi}[teo]{Propiedad}

\theoremstyle{definition}
\newtheorem{ejcio}[teo]{Ejercicio}
\newtheorem{conj}[teo]{Conjetura}
\newtheorem{obs}[teo]{Observación}
\newtheorem{defn}[teo]{Definición}
\newtheorem{ax}[teo]{Axioma}
\newtheorem{ex}[teo]{Ejemplo}

\newcommand{\bd}[1]{\mathbf{#1}}  % for bolding symbols
\newcommand{\cl}[1]{\overline{#1}} 
\newcommand{\CC}{\mathbb{C}}
\newcommand{\RR}{\mathbb{R}}      % for Real numbers
\newcommand{\ZZ}{\mathbb{Z}}      % for Integers
\newcommand{\NN}{\mathbb{N}}
\newcommand{\QQ}{\mathbb{Q}}
\newcommand{\FF}{\mathbb{F}}
\newcommand{\col}[1]{\left[\begin{matrix} #1 \end{matrix} \right]}
\newcommand{\comb}[2]{\binom{#1^2 + #2^2}{#1+#2}}
\newcommand{\eps}{\varepsilon}
\renewcommand{\hom}{\mathrm{Hom}}
\let\oldemptyset\emptyset
\let\emptyset\varnothing
\DeclareMathOperator{\id}{id}
\DeclareMathOperator{\mcm}{mcm}
\DeclareMathOperator{\mcd}{mcd}
\DeclareMathOperator{\ord}{ord}
\DeclareMathOperator{\im}{im}
\DeclareMathOperator{\End}{End}
\DeclareMathOperator{\Aut}{Aut}
\DeclareMathOperator{\sg}{sg}
\DeclareMathOperator{\cok}{cok}
\DeclareMathOperator{\ext}{Ext}
\DeclareMathOperator{\Obj}{Obj}
\DeclareMathOperator{\rank}{rk}
\DeclareMathOperator{\gr}{gr}
\DeclareMathOperator{\car}{char}
\DeclareMathOperator{\Nil}{Nil}
\DeclareMathOperator{\spec}{Spec}
\DeclareMathOperator{\ev}{ev}
\DeclareMathOperator{\ann}{Ann}
\DeclareMathOperator{\tr}{Tr}
\DeclareMathOperator*{\bigcdot}{\scalerel*{\cdot}{\bigodot}}
\def\acts{\curvearrowright}
\def\stca{\curvearrowleft}

\setcounter{tocdepth}{10}
\setcounter{secnumdepth}{10}

\usepackage[utf8]{inputenc}
\usepackage{fancyhdr}


\begin{document}

%----------------------------------------------------------------------------------------
%	TITLE PAGE
%----------------------------------------------------------------------------------------

\begin{titlepage}

\center % Center everything on the page

\newcommand{\HRule}{\rule{\linewidth}{0.5mm}} % Defines a new command for the horizontal lines, change thickness here

\textsc{\LARGE Universidad de Buenos Aires}\\[1.5cm] % Name of your university/college
\textsc{\Large Facultad de Ciencias Exactas y Naturales}\\[0.5cm] % Major heading such as course name
\textsc{\large Departamento de Computación}\\[0.5cm] % Minor heading such as course title
\textsc{\large Algoritmos y Estructuras de Datos III}\\[0.5cm] % Minor heading such as course title

\HRule \\[0.8cm]
{ \huge \bfseries Trabajo Práctico 3}\\[0.4cm] % Title of your document
\HRule \\[1.5cm]

\begin{minipage}{0.8\textwidth}
\center
%\begin{flushleft} 
\Large
\emph{Autores:}


{Nicolás Chehebar, mail: \textit{nicocheh@hotmail.com}, LU: 308/16 

Matías Duran, mail: \textit{mato\_ fede@live.com.ar}, LU: 400/16 

Lucas Somacal, mail: \textit{lsomacal@gmail.com}, LU: 249/16} % Your name
%\end{flushleft}
~
\end{minipage}\\[4cm]

%\includegraphics{Logo}\\[1cm] % Include a department/university logo - this will require the graphicx package

\vfill % Fill the rest of the page with whitespace

\end{titlepage}

%----------------------------------------------------------------------------------------
%	TABLE OF CONTENTS
%----------------------------------------------------------------------------------------


 %\chead{Algo III, TP2, Chehebar, Duran, Somacal}
 
%\title{Algoritmos y Estructuras de Datos III, TP2}
%\author{Nicolás Chehebar, Matías Duran, Lucas Somacal}
%\date{}



\pagenumbering{roman}
\pagenumbering{arabic}
%\maketitle
\tableofcontents
\clearpage







\section{El Juego}
\subsection{Descripción}

\subsubsection{Ejemplos}










\section{1.a) Jugador Óptimo, Minimax sin poda}

\subsection{El algoritmo}

\subsubsection{El Pseudocódigo}



\subsection{Complejidad}


\subsection{Experimentación}









\section{1.b) Jugador Óptimo, Minimax con poda alfa-beta}


\subsection{El algoritmo}

\subsubsection{El Pseudocódigo}



\subsection{Complejidad}


\subsection{Experimentación}

\subsubsection{Contexto} 
\subsubsection{Experimentos}

\section{Análisis comparativo con paper}








\end{document}